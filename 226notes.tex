\documentclass{report}
\usepackage{amsmath}
\usepackage{amsthm}
\usepackage{amssymb}

% Definitions / Theorems
%%%%%%%%%%%%%%%%%%%%%%
\newtheorem{definition}{Definition}
\newtheorem{thm}{Theorem}
%%%%%%%%%%%%%%%%%%%%%%

% Examples
%%%%%%%%%%%%%%%%%%%%%%
\newtheorem*{remark*}{Example}
\newtheorem*{definition*}{Meaning}
%%%%%%%%%%%%%%%%%%%%%%

% Misc
%%%%%%%%%%%%%%%%%%%%%%
\newtheorem{plain}{Symbol}
%%%%%%%%%%%%%%%%%%%%%%

% Real Numbers
%%%%%%%%%%%%%%%%%%%%%%
\newcommand{\R}{\mathbb{R}}
\newcommand{\N}{\mathbb{N}}
%%%%%%%%%%%%%%%%%%%%%%

\title{Math\_226 Notes}
\author{Randy Truong}
\begin{document}
\maketitle
\begin{sloppypar}
  \tableofcontents

% Sequences (Part One)
\chapter{10.1 (Part One): Sequences (Part One) (03/28/23)}
\section{Reminders}
\begin{itemize}
  \item The first MyLab homework is due on
        \textbf{Wednesday, March 28, 2023}.
        \begin{itemize}
                \item Series (Part 0)
        \end{itemize}
  \item The first \textbf{written homework} is going to
        be due on \textbf{Friday, March 31, 2023}.

\end{itemize}

\section{Objectives}
\begin{itemize}
  \item We want to be able to derive the concept of a
        series and a sequence.
  \item We want to be able to understand
        where the idea of a series and sequence come from,
        especially from seemingly-ordinary objects.
  \item We want to explore the idea of a limit
        in relation to a sequence.
  \item We want to be able to discretely express
        a sequence.

\end{itemize}

\section{Motivation}
In former calculus classes, we have observed the idea of limits, derivatives, and integrals at face value. We
know how to evaluate these different calculations, but
what exactly do they mean in the context of math? How can
we better observe what exactly happens in these calculations,
and understand them outside the context of visualizing graphs
or projectile motion.
\section{A third...}

% Sequences (Part Two)
\chapter{10.1 (Part Two): Sequences (Part Two) (03/29/23)}
\section{Reminders}
\begin{itemize}
  \item MyLab Math Assignment 1 - \textbf{Sequences (Part 0)} is
        due \textbf{tonight, March 29, 2023}.
  \item Written Homework 1 is due \textbf{Friday, March 31,
        2023} at the \textbf{beginning of class}.
  \item Friday's lesson is going to be over \textbf{sections
        4.6 and 10.1} and we will be talking about \textbf{
        Newton's Method. }
\end{itemize}
\subsection{Course Philosophy}
Remember that the entire point of this class is to develop
an intuition and a larger understanding and appreciation of
calculus.
\begin{center}
  \textit{``Calculus is just algebra with a tiny drop of
    limits''}
\end{center}
\section{Motivation}
In the last class, we got our first taste of sequences
by exploring the idea of a third and eventually relating it
to the \textbf{geometric sequence}, which results in the formula
of
\[ a_{n} = \frac{1}{1-x}\]
In this class, we were essentially taking our ``preview'' of
sequences, actually defining different aspects of our sequence,
doing operations on sequences, and finally, exploring
one of the most important ideas of sequences, which are
limit convergence and divergence, which led us to the famous
$ \varepsilon - N $ proof, which is also known as the \textbf{precise
definition of convergence.}
\section{Sequences}
\begin{center}
  \fbox {
    \parbox{\textwidth} {
      \begin{definition}
        Sequences
      \end{definition}
      A function with a domain of natural numbers and a co-domain
      of real numbers.
      \[ f: \N \rightarrow \R \]
      \begin{definition*}
      \end{definition*}
      \par Although our intuition would tell us that a sequence
        is just a list or a collection of numbers,
        a sequence is more precisely just a function in which
        we input an \textbf{index} (a natural number) and we
        output a \textbf{term} (a real number). We, of course,
        then, collect these outputs, and this is what we
        generally see.
        \begin{remark*}
          \[ \{ a_{n} \}\]
          \[ \{ a_{n}\}_{n=1}^{\infty}\]
          \[ \{ a_{n}\}_{n=0}^{\infty}\]
          \[ 1, 2, 3, 4, \dots\]
          \[ 1.1, 2.2, 3.3, 4.4, \dots \]

          \end{remark*}

    }}
\end{center}

\section{Convergence}
\begin{center}
  \fbox{
    \parbox{\textwidth}{
      \begin{definition}
        Sequence Convergence
      \end{definition}
      \textbf{Informal Definition.}
      \par A sequence $ \{ a_{n }\} $ converges to a limit $ L $
      if the terms get arbitrarily close to $ L $ as $ n $
      gets sufficiently large, which is also known as

      \[ \lim_{n\rightarrow\infty}a_{n} = L \]
      \par
      \textbf{Formal Definition.}
      \par A sequence $ \{ a_{n}\} $ converges to a limit $ L $
      if, for every $ \varepsilon > 0 $, where $ \varepsilon $ is the distance from
      the range to the limit $ L $, there exists such a number
      that
      \[ |a_{n}- L| < \varepsilon ~\textrm{for}~ n \geq N \]

      The preceding expression is also known as the $ \varepsilon - N $ proof.
      \[ \]
    }}
\end{center}

\section{Divergence}
\begin{center}
  \fbox{
    \parbox{\textwidth}{
      \begin{definition}
        Sequence Divergence
      \end{definition}
      \textbf{Informal Definition.}
      \par
      A sequence diverges when it doesn't converge. If the sequence
      $ \{ a_{n }\} $ does not get arbitrarily close to limit $ L $
      as $ n $ gets sufficiently large. This is also known as
      when the limit $ L $ \textbf{does not exist.}
      \par
      \[ \lim_{n\rightarrow\infty}a_{n} ~ \textrm{does not exist}\]
      \textbf{Formal Definition.}
      \par
      A sequence $ { a_{n} }$ diverges to (positive) infinity if, for every
      $ M > 0 $, there exists an N such that
      \[ a_{n} > M ~ \textrm{whenever}~ n > M \]

      Additionally, a sequence $ { a_{n} } $ diverges to negative
      infinity, if, for every $ M < 0 $, there exists an $ N $
      such that
      \[ a_{n} < M ~ \textrm{whenever} ~ n > M \]

    }
  }
\end{center}

\section{$\varepsilon-N$ Proof}
In this proof, we are proving the existence of a limit
when given a sequence $a_{n}$
\section{Properties of Sequence Limits}

% Sequences (Part Three)
\chapter{10.1: Sequences (Part 3) (03/31/23)}
\section{Summary of Lesson}
In this lesson, we further develop the idea of convergence
and divergence by \textbf{expanding it to non-elementary sequences}. We
apply several new theorems as a result of this, such as the
Sandwich Theorem. The homework, as a result, is all about
convergence and divergence, testing on how well we are
able to determine convergence and divergence given sequences
including factorials, logarithms, and exponential functions.
\section{Reminders}
\begin{itemize}
  \item The \textbf{third MyLab Math assignment} (Sequences
        (Part 2)) is due on \textbf{Tuesday, April 3rd}.
  \item The second written assignment is due on \textbf{
        Friday, April 7, 2023}.
\end{itemize}
\section{Motivation}


In the previous lesson, we learned about the \textbf{precise
  definition of convergence and divergence}. We learned about
what exactly convergence and divergence means, and we were
able to apply these concepts to various elementary sequences.
However, what if we were given a sequence like
\[ \{a_{n}\} = \frac{\cos(n)}{n}\]
Additionally, what if we were given a sequence like
\[ \{a_{n}\} = \Biggr( 1 + \frac{x}{n} \Biggr)^{n}\]

These sequences are undoubtedly more complex than our former
examples and are not nearly as intuitive whenever it comes to
actually solving them.
\par Therefore, we need to learn a few more \textbf{theorems} as
well as \textbf{techniques} in order to determine convergence
and divergence in more complex sequence functions.
\begin{center}
  \fbox{
    \parbox{\textwidth}{
      \begin{thm}
        Sandwich Theorem
      \end{thm}
      Let $ \{a_{n}\}$, $ \{b_{n}\}$, $ \{ c_{n}\}$ be sequences
      of real numbers. If $ a_{n} \leq b_{n} \leq c_{n} $ holds for all
      $ n $ beyond some index $ N $, and if $ \lim_{n \rightarrow \infty }a_{n} =
      \lim_{n \rightarrow \infty } c_{n}= L $, then $ \lim_{n \rightarrow L } b_{n} = L $ also.
      \[ \textrm{let}~ \{a_{n}\},\{b_{n}\}, \{c_{n}\} ~ \textrm{
          be sequences
        }\]
      \[ \textrm{if}~ a_{n} \leq b_{n} \leq c_{n}~\textrm{and}  \]
      \[ \lim_{n \rightarrow \infty}a_{n} = L~\textrm{and}~\lim_{n \rightarrow \infty}c_{n}= L,~
        \textrm{then}\]
      \[ \lim_{n \rightarrow \infty} b_{n} = L\]
    }
  }
\end{center}
\begin{center}
  \fbox{
    \parbox{\textwidth}{
      \begin{thm}
        The Continuous Function Theorem For Sequences
      \end{thm}

    }
  }
\end{center}
\begin{center}
  \fbox{
    \parbox{\textwidth}{
      \begin{thm}
        Other Theorem
      \end{thm}

    }
  }
\end{center}

\begin{center}
  \fbox{
    \parbox{\textwidth}{
      Theorem 5
    }
  }
\end{center}




\chapter{4.6, 10.1: Finishing Sequences +  Newton's Method (04/03/2023)}
\section{Reminders}
\begin{itemize}
  \item The \textbf{third MyLab Math assignment} (Sequences
        (Part 2)) is due on \textbf{Tuesday, April 3rd}.
  \item The second written assignment is due on \textbf{
        Friday, April 7, 2023}.
\end{itemize}
\section{Motivation}
\chapter{10.2 (Part One): Infinite Series (Part One)}
\section{Reminders}
\begin{itemize}
  \item The \textbf{third MyLab Math assignment} (Sequences
        (Part 2)) is due on \textbf{Tuesday, April 3rd}.
  \item The second written assignment is due on \textbf{
        Friday, April 7, 2023}.
\end{itemize}

\section{Motivation}
\chapter{10.2 (Part Two): Infinite Series (Part Two) (04/05/23)}
\section{Reminders}
\begin{itemize}
  \item The \textbf{third MyLab Math assignment} (Sequences
        (Part 2)) is due on \textbf{Tuesday, April 3rd}.
  \item The second written assignment is due on \textbf{
        Friday, April 7, 2023}.
\end{itemize}

\section{Motivation}
\chapter{10.3: The Integral Test (04/07/23)}
\section{Reminders}
\section{Motivation}
\chapter{10.4: Comparision Tests (04/10/23)}
\section{Reminders}
\section{Motivation}
\chapter{10.5: Absolute Convergence and the Ratio Test}
\section{Reminders}
\section{Motivation}
\chapter{10.6: Alternating Series and Conditional Convergence}
\section{Reminders}
\section{Motivation}
\chapter{10.6: Strategies for Analyzing Convergence}
\section{Reminders}
\section{Motivation}
%%%%%%%%%%%% MIDTERM 1 DONE %%%%%%%%%%%%%%%%%%%%%%%%%
\chapter{10.7 (Part One): Power Series}
\section{Reminders}
\section{Motivation}
\chapter{10.7 (Part Two): Radius and Interval of Convergence}
\section{Reminders}
\section{Motivation}
\chapter{10.7 (Part Three): Manipulation of Series (Part One)}
\section{Reminders}
\section{Motivation}
\chapter{10.7 (Part Four): Manipulation of Series (Part Two)}
\section{Reminders}
\section{Motivation}
\chapter{10.8 (Part One):}
\section{Reminders}
\section{Motivation}
\chapter{10.8 (Part Two):}
\section{Reminders}
\section{Motivation}
\chapter{10.9: Convergence of Taylor Series}
\section{Reminders}
\section{Motivation}
\chapter{10.10: Applications of Taylor Series}
\section{Reminders}
\section{Motivation}
\chapter{A7 (Part One): Complex Numbers (Part One)}
\section{Reminders}
\section{Motivation}
\chapter{10.10, A7 (Part Two): Complex Numbers (Part Two)}
\section{Reminders}
\section{Motivation}
\chapter{19.1 (Part One): Vectors}
\section{Reminders}
\section{Motivation}
\chapter{19.1 (Part Two): Inner Products}
\section{Reminders}
\section{Motivation}
%%%%%%%%%%%% MIDTERM 2 DONE %%%%%%%%%%%%%%%%%%%%%%%%%

\chapter{19.2 (Part One): Functions as Vectors, Periodic
  Functions}
\section{Reminders}
\section{Motivation}
\chapter{19.2 (Part Two): Fourier Series, Demos}
\section{Reminders}
\section{Motivation}
\chapter{19.3: Fourier Series, Theory}
\section{Reminders}
\section{Motivation}
\chapter{19.5 (Part One): Applications (Part One)}
\section{Reminders}
\section{Motivation}
\chapter{19.5 (Part Two): Applications (Part Two)}
\section{Reminders}
\section{Motivation}
\end{sloppypar}

\end{document}
