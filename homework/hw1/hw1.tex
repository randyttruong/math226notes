\documentclass{article}
\usepackage{amsmath}
\usepackage{amsthm}
\usepackage{amssymb}

% Definitions / Theorems
%%%%%%%%%%%%%%%%%%%%%%
\newtheorem{definition}{Definition}
\newtheorem{thm}{Theorem}
%%%%%%%%%%%%%%%%%%%%%%

% Examples
%%%%%%%%%%%%%%%%%%%%%%
\newtheorem*{remark*}{Example}
\newtheorem*{definition*}{Meaning}
%%%%%%%%%%%%%%%%%%%%%%

% Misc
%%%%%%%%%%%%%%%%%%%%%%
\newtheorem{plain}{Symbol}
\newenvironment{problem}[2][Problem]{\begin{trivlist}
\item[\hskip \labelsep {\bfseries #1}\hskip \labelsep {\bfseries #2.}]}{\end{trivlist}}

%%%%%%%%%%%%%%%%%%%%%%

% Real Numbers
%%%%%%%%%%%%%%%%%%%%%%
\newcommand{\R}{\mathbb{R}}
\newcommand{\N}{\mathbb{N}}
%%%%%%%%%%%%%%%%%%%%%%

\title{Math\_226 Written Homework 1: Sequences}
\author{Randy Truong}
\begin{document}
\maketitle
\begin{problem}{1.1}
  Compute $ | a_{n} - 2 | $, simplifying your answer as much
  as possible.
\end{problem}
\[ \textrm{let $ a_{n} $} = \frac{2n+5}{n+1}\]
\[ \Rightarrow |a_{n} - 2 |\]
\[ \Rightarrow \Biggr|\frac{2n+5}{n+1} - 2 \Biggr|\]
\[ \Rightarrow \Biggr|\frac{2n+5}{n+1} - \frac{2(n+1)}{n+1}\Biggr| \]
\[ \Rightarrow \Biggr|\frac{3}{n+1}\Biggr| = |a_{n} - 2|\]

\begin{problem}{1.2}
  Take $ r = 1 $ in the definition of convergence. Find
  the first value of $ N $ which satisfies $ |a_{N} - 2 | < 1 $.
  \\
  \\
  \textbf{Recall.}
  \[ |a_{n} - L| < r ~\textrm{such that}~ n>N \]

  \[ \textrm{let r}~ = 1, ~ \textrm{let} ~|a_{N} - 2| = \Biggr|\frac{3}{n+1}\Biggr|\]
  \[ \Rightarrow |a_{N} - 2| < 1\]
  \[ \Rightarrow \Biggr| \frac{3}{n+1} \Biggr| < 1 \]
  \[ \Rightarrow -1 < \frac{3}{n+1} < 1 \]
  \[ \Rightarrow -1(n+1) < 3 < 1(n+1) \]
  \[ \Rightarrow -n - 1 < 3 < n + 1 \]
  \[ \Rightarrow -n -1 < 3 ~\textrm{and} ~ 3 < n+1 \]
  \[ \Rightarrow -n < 4 ~\textrm{and} ~ 2 < n \]
  \[ \Rightarrow n > -4 ~ \textrm{and} ~ n > 2 \]

  Because we know that $ N \in \N $, then $ N $ cannot be $ -4 $.
  Therefore, we know that $ N > 2 $. \\

  \begin{center}
    The first value $ N $ that satisfies $ |a_{n} - 2| < 1 $ must
    be 3.
  \end{center}

  \[ \]

\end{problem}
\begin{problem}{1.3}
  Justify the fact that the sequence $ a_{n} $ is decreasing. \\

    Let us consider the derivative of $ a_{n} $. By considering
    the derivative, we are able to observe the relationship
    between the indices $ n $ of the sequence and the terms $ a_{n} $, and, most importantly,
    whether or not $ a_{n} $ increases or decreases.
    \[ \textrm{let $ a_{n} = f(x)$}\]
    \begin{center}
      for all $ x $ where $ x $ is a real number
    \end{center}

    \[ \textrm{$ f(x) $} = \frac{2x+5}{x+1} \]

    \[ \Rightarrow f'(x) = \frac{d}{dx} \Biggr(\frac{2x+5}{x+1} \Biggr)\]

    \[ \Rightarrow \Biggr( \frac{(x+1)(2x+5)' - (2x+5)(x+1)'}{(x+1)^{2}} \Biggr)\]
    \[ \Rightarrow \Biggr( \frac{(x+1)(2) - (2x+5)(1)}{(x+1)^{2}}\Biggr)\]
    \[ \Rightarrow \Biggr( \frac{2x+2 - 2x - 5}{(x+1)^{2}}\Biggr)\]
    \[ \Rightarrow -\frac{3}{(x+1)^{2}} = f'(x) \]

    Given that the derivative $ f'(x) = -\frac{3}{(x+1)^{2}}$,
    we can see that the value of $ f'(x) < 0 $ for all $ x $.
    Therefore, the range $ a_{n} $ of the function $ a_{n} = f(n) $ decreases as
    the domain $ n$ increases.

\end{problem}
\begin{problem}{1.4}
  Justify how all terms in the sequence beyond $ a_{N} $
  also satisfy the definition
  \[ |a_{n} - 2| < 1 \]

  \textbf{Solution.}
  \\
  By solving for the first term $ a_{n}$ that satisfies
  the  $ n > 2 $, $ a_{3} $,  we observe
  \[ a_{3} = \frac{2(3)+5}{3+1} = \frac{11}{4} < (2+1) \]
  \[ \Rightarrow (2-1) < \frac{11}{4} < (2+1) \]

  Therefore,
  \[ \Rightarrow \Biggr|\frac{11}{4} - 2\Biggr| < 1 \]
  \[ \Rightarrow  |a_{3} - 2| < 1 \]
  \\
  Given that $ |a_{3} - 2| < 1 $ and that, from question (3),  the derivative  $ f'(x)$ decreases, all terms $a_{n}$ subsequent  $n > 2 $ and $ n = 3 $ must decrease and also be less than $ 1$.
  \[ |a_{n} - 2| < |a_{3} - 2| < 1 \]
  In addition, we must also observe that for all terms $ a_{n} $ in the sequence  where $ n > 2 $,  the difference $ |a_{n}-2| $ will always
  be greater than 0, since we know that the limit 2 will always
  be less than the function $ a_{n} $ for all $ n$
  \[ 2 = \frac{2(n+1)}{n+1} < \frac{2n+5}{n+1}\]

  We can
  conclude that all of the terms $a_{n}$ following $ n = 3 $ will
  lie  within the bounds (2-1, 2+1), since their distance
  will always be less than 1 but will never be less than 0.
  \[ 0 < |a_{n} - 2| < |a_{3} - 2| < 1 \]
  \[ \Rightarrow 0 < |a_{n} - 2| < |a_{3} - 2| < 1 \]
\end{problem}
\begin{problem}{1.5.1}
  Repeat Questions (1-4) for $ r = \frac{1}{100} $
  \[ \textrm{let}~r =\frac{1}{100}  \]
  \[ \textrm{let}~(a_{n}-2) = \frac{3}{n+1}\]
  \[ |a_{n}-2| < r\]
  \[ \Rightarrow \Biggr| \frac{3}{n+1} \Biggr| < \frac{1}{100}\]
  \[ \Rightarrow -\frac{1}{100} < \frac{3}{n+1} < \frac{1}{100} \]
  \[ \Rightarrow -n - 1 < 300 ~\textrm{and}~ n+1 > 300 \]
  \[ \Rightarrow n > -301 ~\textrm{and}~ n > 299 \]
  Because $ n \in \N $, we omit all negative numbers. Therefore
  \[ N > 299 \]
   By solving for the first term $ a_{n}$ that satisfies
   $ N > 299 $, $ a_{300} $, we observe
  \[ a_{300} = \frac{2(300)+5}{300+1} = \frac{605}{301} = 2.009967... < (2+\frac{1}{100}) \]
  \[ \Rightarrow (2-\frac{1}{100}) < \frac{605}{301} < (2+\frac{1}{100}) \]

  Therefore,
  \[ \Rightarrow \Biggr|\frac{605}{301} - 2\Biggr| < \frac{1}{100} \]
  \[ \Rightarrow  |a_{300} - 2| < \frac{1}{100} \]
  \\
  Note that we do not need to calculate the derivative of the function $ a_{n} = f(x) $, since the function itself remains the same.
  \\
  \\
  Given that $ |a_{300} - 2| < \frac{1}{100} $ and that, from question (3),  the derivative  $ f'(x)$ decreases, all terms $a_{n}$ subsequent  $n > 299 $ and $ n = 300 $ must decrease and in turn be less than $ \frac{1}{100}$.
  \[ |a_{n} - 2| < |a_{300} - 2| < \frac{1}{100} \]
  In addition, we must also observe that for all terms $ a_{n} $ in the sequence  where $ n > 299 $,  the difference $ |a_{n}-2| $ will always
  be greater than 0, since we know that the limit 2 will always
  be less than the function $ a_{n} $ for all $ n$
  \[ 2 = \frac{2(n+1)}{n+1} < \frac{2n+5}{n+1}\]

  We can
  conclude that all of the terms $a_{n}$ following $ n = 300 $ will
  lie  within the bounds $(2-\frac{1}{100}, 2+\frac{1}{100})$, since their distance
  will always be less than $\frac{1}{100}$ but will never be less than 0.
  \[ 0 < |a_{n} - 2| < |a_{300} - 2| < \frac{1}{100}\]
  \[ \Rightarrow 0 < |a_{n} - 2| < |a_{300} - 2| < \frac{1}{100} \]
\end{problem}
\begin{problem}{1.5.2}
  Repeat Questions (1-4) for $ r = \frac{1}{1000}$
    \[ \textrm{let}~r =\frac{1}{1000}  \]
    \[ \textrm{let}~(a_{n}-2) = \frac{3}{n+1}\]
    \[ |a_{n}-2| < r \]
  \[ \Rightarrow \Biggr| \frac{3}{n+1} \Biggr| < \frac{1}{1000}\]
  \[ \Rightarrow -\frac{1}{1000} < \frac{3}{n+1} < \frac{1}{1000} \]
  \[ \Rightarrow -n - 1 < 3000 ~\textrm{and}~ n+1 > 3000 \]
  \[ \Rightarrow n > -3001 ~\textrm{and}~ n > 2999 \]
  Because $ n \in \N $, we omit all negative numbers. Therefore
  \[ N > 2999 \]
  By solving for the first term $ a_{n}$ that satisfies
  the  $ N > 2999 $, $ a_{3000} $, we observe
  \[ a_{3000} = \frac{2(3000)+5}{3000+1} = \frac{6005}{3001} =
    2.000999... < (2+\frac{1}{1000}) \]
  \[ \Rightarrow (2-\frac{1}{1000}) < \frac{6005}{3001} < (2+\frac{1}{1000}) \]

  Therefore,
  \[ \Rightarrow \Biggr|\frac{6005}{3001} - 2\Biggr| < \frac{1}{1000} \]
  \[ \Rightarrow  |a_{3000} - 2| < \frac{1}{1000} \]
  Again, we do not need to re-calculate the derivative of $ a_{n} = f(x)$, since the function does not change.
  \\
  \\
  Given that $ |a_{3000} - 2| < \frac{1}{1000} $ and that, from question (3),  the derivative  $ f'(x)$ decreases, all terms $a_{n}$ subsequent  $n > 2999 $ and $ n = 3000 $ must decrease and in turn be less than $ \frac{1}{1000}$.
  \[ |a_{n} - 2| < |a_{3000} - 2| < \frac{1}{1000} \]
  In addition, we must also observe that for all terms $ a_{n} $ in the sequence  where $ n > 2999 $,  the difference $ |a_{n}-2| $ will always
  be greater than 0, since we know that the limit 2 will always
  be less than the function $ a_{n} $ for all $ n$
  \[ 2 = \frac{2(n+1)}{n+1} < \frac{2n+5}{n+1}\]

  We can
  conclude that all of the terms $a_{n}$ following $ n = 3000 $ will
  lie  within the bounds $(2-\frac{1}{1000}, 2+\frac{1}{1000})$, since their distance
  will always be less than $ \frac{1}{1000}$ but will never be less than 0.
  \[ 0 < | a_{n} - 2| < |a_{3000} - 2| < \frac{1}{1000} \]
  \[ \Rightarrow 0 < |a_{n} - 2| < |a_{3000} - 2| < \frac{1}{1000} \]

\end{problem}


\end{document}
[]
